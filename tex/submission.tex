% This contents of this file will be inserted into the _Solutions version of the
% output tex document.  Here's an example:

% If assignment with subquestion (1.a) requires a written response, you will
% find the following flag within this document: <SCPD_SUBMISSION_TAG>_1a
% In this example, you would insert the LaTeX for your solution to (1.a) between
% the <SCPD_SUBMISSION_TAG>_1a flags.  If you also constrain your answer between the
% START_CODE_HERE and END_CODE_HERE flags, your LaTeX will be styled as a
% solution within the final document.

% Please do not use the '<SCPD_SUBMISSION_TAG>' character anywhere within your code.  As expected,
% that will confuse the regular expressions we use to identify your solution.
\def\assignmentnum{2 }
\def\assignmenttitle{XCS229i Problem Set \assignmentnum}

\documentclass{article}
\usepackage[top = 1.0in]{geometry}

\usepackage{graphicx}

\usepackage[utf8]{inputenc}
\usepackage{listings}
\usepackage[dvipsnames]{xcolor}
\usepackage{bm}
\usepackage{algorithm}
\usepackage{algpseudocode}
\usepackage{framed}

\definecolor{codegreen}{rgb}{0,0.6,0}
\definecolor{codegray}{rgb}{0.5,0.5,0.5}
\definecolor{codepurple}{rgb}{0.58,0,0.82}
\definecolor{backcolour}{rgb}{0.95,0.95,0.92}

\lstdefinestyle{mystyle}{
    backgroundcolor=\color{backcolour},   
    commentstyle=\color{codegreen},
    keywordstyle=\color{magenta},
    stringstyle=\color{codepurple},
    basicstyle=\ttfamily\footnotesize,
    breakatwhitespace=false,         
    breaklines=true,                 
    captionpos=b,                    
    keepspaces=true,                 
    numbersep=5pt,                  
    showspaces=false,                
    showstringspaces=false,
    showtabs=false,                  
    tabsize=2
}

\lstset{style=mystyle}

\newcommand{\di}{{d}}
\newcommand{\nexp}{{n}}
\newcommand{\nf}{{p}}
\newcommand{\vcd}{{\textbf{D}}}
\newcommand{\Int}{\mathbb{Z}}
\newcommand\bb{\ensuremath{\mathbf{b}}}
\newcommand\bs{\ensuremath{\mathbf{s}}}
\newcommand\bp{\ensuremath{\mathbf{p}}}
\newcommand{\relu} { \operatorname{ReLU} }
\newcommand{\smx} { \operatorname{softmax} }
\newcommand\bx{\ensuremath{\mathbf{x}}}
\newcommand\bh{\ensuremath{\mathbf{h}}}
\newcommand\bc{\ensuremath{\mathbf{c}}}
\newcommand\bW{\ensuremath{\mathbf{W}}}
\newcommand\by{\ensuremath{\mathbf{y}}}
\newcommand\bo{\ensuremath{\mathbf{o}}}
\newcommand\be{\ensuremath{\mathbf{e}}}
\newcommand\ba{\ensuremath{\mathbf{a}}}
\newcommand\bu{\ensuremath{\mathbf{u}}}
\newcommand\bv{\ensuremath{\mathbf{v}}}
\newcommand\bP{\ensuremath{\mathbf{P}}}
\newcommand\bg{\ensuremath{\mathbf{g}}}
\newcommand\bX{\ensuremath{\mathbf{X}}}
% real numbers R symbol
\newcommand{\Real}{\mathbb{R}}

% encoder hidden
\newcommand{\henc}{\bh^{\text{enc}}}
\newcommand{\hencfw}[1]{\overrightarrow{\henc_{#1}}}
\newcommand{\hencbw}[1]{\overleftarrow{\henc_{#1}}}

% encoder cell
\newcommand{\cenc}{\bc^{\text{enc}}}
\newcommand{\cencfw}[1]{\overrightarrow{\cenc_{#1}}}
\newcommand{\cencbw}[1]{\overleftarrow{\cenc_{#1}}}

% decoder hidden
\newcommand{\hdec}{\bh^{\text{dec}}}

% decoder cell
\newcommand{\cdec}{\bc^{\text{dec}}}

\usepackage[hyperfootnotes=false]{hyperref}
\hypersetup{
  colorlinks=true,
  linkcolor = blue,
  urlcolor  = blue,
  citecolor = blue,
  anchorcolor = blue,
  pdfborderstyle={/S/U/W 1}
}
\usepackage{nccmath}
\usepackage{mathtools}
\usepackage{graphicx,caption}
\usepackage[shortlabels]{enumitem}
\usepackage{epstopdf,subcaption}
\usepackage{psfrag}
\usepackage{amsmath,amssymb,epsf}
\usepackage{verbatim}
\usepackage{cancel}
\usepackage{color,soul}
\usepackage{bbm}
\usepackage{listings}
\usepackage{setspace}
\usepackage{float}
\definecolor{Code}{rgb}{0,0,0}
\definecolor{Decorators}{rgb}{0.5,0.5,0.5}
\definecolor{Numbers}{rgb}{0.5,0,0}
\definecolor{MatchingBrackets}{rgb}{0.25,0.5,0.5}
\definecolor{Keywords}{rgb}{0,0,1}
\definecolor{self}{rgb}{0,0,0}
\definecolor{Strings}{rgb}{0,0.63,0}
\definecolor{Comments}{rgb}{0,0.63,1}
\definecolor{Backquotes}{rgb}{0,0,0}
\definecolor{Classname}{rgb}{0,0,0}
\definecolor{FunctionName}{rgb}{0,0,0}
\definecolor{Operators}{rgb}{0,0,0}
\definecolor{Background}{rgb}{0.98,0.98,0.98}
\lstdefinelanguage{Python}{
    numbers=left,
    numberstyle=\footnotesize,
    numbersep=1em,
    xleftmargin=1em,
    framextopmargin=2em,
    framexbottommargin=2em,
    showspaces=false,
    showtabs=false,
    showstringspaces=false,
    frame=l,
    tabsize=4,
    % Basic
    basicstyle=\ttfamily\footnotesize\setstretch{1},
    backgroundcolor=\color{Background},
    % Comments
    commentstyle=\color{Comments}\slshape,
    % Strings
    stringstyle=\color{Strings},
    morecomment=[s][\color{Strings}]{"""}{"""},
    morecomment=[s][\color{Strings}]{'''}{'''},
    % keywords
    morekeywords={import,from,class,def,for,while,if,is,in,elif,else,not,and,or
    ,print,break,continue,return,True,False,None,access,as,,del,except,exec
    ,finally,global,import,lambda,pass,print,raise,try,assert},
    keywordstyle={\color{Keywords}\bfseries},
    % additional keywords
    morekeywords={[2]@invariant},
    keywordstyle={[2]\color{Decorators}\slshape},
    emph={self},
    emphstyle={\color{self}\slshape},
%
}
\lstMakeShortInline|

\pagestyle{empty} \addtolength{\textwidth}{1.0in}
\addtolength{\textheight}{0.5in}
\addtolength{\oddsidemargin}{-0.5in}
\addtolength{\evensidemargin}{-0.5in}
\newcommand{\ruleskip}{\bigskip\hrule\bigskip}
\newcommand{\nodify}[1]{{\sc #1}}
\newenvironment{answer}{\sf \begingroup\color{ForestGreen}}{\endgroup}%

\setlist[itemize]{itemsep=2pt, topsep=0pt}
\setlist[enumerate]{itemsep=6pt, topsep=0pt}

\setlength{\parindent}{0pt}
\setlength{\parskip}{4pt}
\setlist[enumerate]{parsep=4pt}
\setlength{\unitlength}{1cm}

\renewcommand{\Re}{{\mathbb R}}
\newcommand{\R}{\mathbb{R}}
\newcommand{\what}[1]{\widehat{#1}}

\renewcommand{\comment}[1]{}
\newcommand{\mc}[1]{\mathcal{#1}}
\newcommand{\half}{\frac{1}{2}}

\DeclareMathOperator*{\argmin}{arg\,min}

\def\KL{D_{KL}}
\def\xsi{x^{(i)}}
\def\ysi{y^{(i)}}
\def\zsi{z^{(i)}}
\def\E{\mathbb{E}}
\def\calN{\mathcal{N}}
\def\calD{\mathcal{D}}
\def\slack{\url{http://xcs229-scpd.slack.com/}}
\def\zipscriptalt{\texttt{python zip\_submission.py}}
\DeclarePairedDelimiter\abs{\lvert}{\rvert}%
 
\usepackage{bbding}
\usepackage{pifont}
\usepackage{wasysym}
\usepackage{amssymb}
\usepackage{framed}
\usepackage{scrextend}

\newcommand{\alns}[1] {
	\begin{align*} #1 \end{align*}
}

\newcommand{\pd}[2] {
 \frac{\partial #1}{\partial #2}
}
\renewcommand{\Re} { \mathbb{R} }
\newcommand{\btx} { \mathbf{\tilde{x}} }
\newcommand{\bth} { \mathbf{\tilde{h}} }
\newcommand{\sigmoid} { \operatorname{\sigma} }
\newcommand{\CE} { \operatorname{CE} }
\newcommand{\byt} { \hat{\by} }
\newcommand{\yt} { \hat{y} }

\newcommand{\oft}[1]{^{(#1)}}
\newcommand{\fone}{\ensuremath{F_1}}

\newcommand{\ac}[1]{ {\color{red} \textbf{AC:} #1} }
\newcommand{\ner}[1]{\textbf{\color{blue} #1}}
\begin{document}
\pagestyle{myheadings} \markboth{}{\assignmenttitle}

% <SCPD_SUBMISSION_TAG>_entire_submission

This handout includes space for every question that requires a written response.
Please feel free to use it to handwrite your solutions (legibly, please).  If
you choose to typeset your solutions, the |README.md| for this assignment includes
instructions to regenerate this handout with your typeset \LaTeX{} solutions.
\ruleskip

\LARGE
1.a
\normalsize

% <SCPD_SUBMISSION_TAG>_1a
\begin{answer}
  Since $g'(z) = g(z)(1-g(z))$ and $h(x) = g(\theta^T x)$, it follows that $\partial h(x) / \partial \theta_k = h(x)(1 - h(x)) x_k$.

  Letting $h_{\theta}(x^{(i)}) = g(\theta^T x^{(i)})
  = 1/(1 + \exp(-\theta^T x^{(i)})$, we have\\

  \begin{flalign*}
    \frac{\partial\log h_{\theta}(x^{(i)})}{\partial\theta_k} &= \\
    % ### START CODE HERE ###
    % ### END CODE HERE ###
    \frac{\partial\log(1 - h_{\theta}(x^{(i)}))}{\partial\theta_k} &= \\
    % ### START CODE HERE ###
    % ### END CODE HERE ###
  \end{flalign*}

  Substituting into our equation for $J(\theta)$, we have
  %
  \begin{flalign*}
    \frac{\partial J(\theta)}{\partial\theta_k} &=\\
    % ### START CODE HERE ###
    % ### END CODE HERE ###
  \end{flalign*}
  
  Consequently, the $(k, l)$ entry of the Hessian is given by
  
  \begin{flalign*}
    H_{kl} = \frac{\partial^2 J(\theta)}{\partial\theta_k\partial\theta_l} &=\\
    % ### START CODE HERE ###
    % ### END CODE HERE ###
  \end{flalign*}
  
  Using the fact that $X_{ij} = x_i x_j$ if and only if $X = xx^T$, we have
  
  \begin{flalign*}
    H &= \\
    % ### START CODE HERE ###
    % ### END CODE HERE ###
  \end{flalign*}

  To prove that $H$ is positive semi-definite, show $z^T Hz \ge 0$ for all $z\in\Re^\di$.
  
  \begin{flalign*}
    z^T H z &=\\
    % ### START CODE HERE ###
    % ### END CODE HERE ###
  \end{flalign*}
  
  % ### START CODE HERE ###
  % ### END CODE HERE ###
\end{answer}
% <SCPD_SUBMISSION_TAG>_1a
\clearpage

\LARGE
1.c
\normalsize

% <SCPD_SUBMISSION_TAG>_1c
\begin{answer}
  For shorthand, we let $\mc{H} = \{\phi, \Sigma, \mu_{0}, \mu_1\}$ denote
  the parameters for the problem.
  Since the given formulae are conditioned on $y$, use Bayes rule to get:
  \begin{align*}
    p(y =1\vert  x ; \mc{H}) &= \frac {p(x\vert y=1; \mc{H}) p(y=1; \mc{H})} {p(x; \mc{H})}\\
    & = \frac {p(x\vert y=1; \mc{H}) p(y=1; \mc{H})}
      {p(x\vert y=1; \mc{H}) p(y=1; \mc{H}) + p(x\vert y={0}; \mc{H}) p(y={0};
      \mc{H})}\\
    &=\\
    % ### START CODE HERE ###
    % ### END CODE HERE ###
  \end{align*}
\end{answer}
% <SCPD_SUBMISSION_TAG>_1c
\clearpage

\LARGE
1.d
\normalsize

% <SCPD_SUBMISSION_TAG>_1d
\begin{answer}
  First, derive the expression for the log-likelihood of the training data:
  \begin{flalign*}
    \ell(\phi, \mu_{0}, \mu_1, \Sigma) &= \log \prod_{i=1}^\nexp p(x^{(i)} \vert  y^{(i)}; \mu_{0}, \mu_1, \Sigma) p(y^{(i)}; \phi)\\
    &= \sum_{i=1}^{\nexp} \log p(x^{(i)} \vert  y^{(i)}; \mu_{0}, \mu_1, \Sigma) +
    \sum_{i=1}^{n} \log p(y^{(i)}; \phi)\\
    &=\\
    % ### START CODE HERE ###
    % ### END CODE HERE ###
  \end{flalign*}

  Now, the likelihood is maximized by setting the derivative (or gradient) with respect to each of the parameters to zero.\\

  \textbf{For $\mathbf{\phi}$:}

  \begin{flalign*}
    \frac{\partial \ell}{\partial \phi}&=\\
    % ### START CODE HERE ###
    % ### END CODE HERE ###
  \end{flalign*}

  Setting this equal to zero and solving for $\phi$ gives the maximum
  likelihood estimate.\\

  \textbf{For $\mathbf{\mu_0}$:}

  {\bf Hint:}  Remember that $\Sigma$ (and thus $\Sigma^{-1}$) is symmetric.

  \begin{flalign*}
    \nabla_{\mu_{0}}\ell &=\\
    % ### START CODE HERE ###
    % ### END CODE HERE ###
  \end{flalign*}

  Setting this gradient to zero gives the maximum likelihood estimate
  for $\mu_{0}$.\\

  \textbf{For $\mathbf{\mu_1}$:}

  {\bf Hint:}  Remember that $\Sigma$ (and thus $\Sigma^{-1}$) is symmetric.

  \begin{flalign*}
    \nabla_{\mu_{1}}\ell &=\\
    % ### START CODE HERE ###
    % ### END CODE HERE ###
  \end{flalign*}

  Setting this gradient to zero gives the maximum likelihood estimate
  for $\mu_{1}$.\\

  For $\Sigma$, we find the gradient with respect to $S = \Sigma^{-1}$ rather than $\Sigma$ just to simplify the derivation (note that $\vert S\vert  = \frac{1}{\vert \Sigma\vert }$).
  You should convince yourself that the maximum likelihood estimate $S_\nexp$ found in this way would correspond to the actual maximum likelihood estimate $\Sigma_\nexp$ as $S_\nexp^{-1} = \Sigma_\nexp$.

  {\bf Hint:}  You may need the following identities: 
  \begin{equation*}
    \nabla_S \vert S\vert  = \vert S\vert  (S^{-1})^T
  \end{equation*}
  \begin{equation*}
    \nabla_S b_i^T S b_i = \nabla_S tr \left( b_i^T S b_i \right) =
    \nabla_S tr \left( S b_i b_i^T \right) = b_i b_i^T
  \end{equation*}

  \begin{flalign*}
    \nabla_S\ell &=\\
    % ### START CODE HERE ###
  % ### END CODE HERE ###
  \end{flalign*}

  Next, substitute $\Sigma = S^{-1}$.  Setting this gradient to zero gives the required maximum likelihood estimate for $\Sigma$.\\
\end{answer}
% <SCPD_SUBMISSION_TAG>_1d
\clearpage

\LARGE
1.f
\normalsize

% <SCPD_SUBMISSION_TAG>_1f
\begin{answer}
  % ### START CODE HERE ###
  % ### END CODE HERE ###
\end{answer}
% <SCPD_SUBMISSION_TAG>_1f
\clearpage

\LARGE
1.g
\normalsize

% <SCPD_SUBMISSION_TAG>_1g
\begin{answer}
  % ### START CODE HERE ###
  % ### END CODE HERE ###
\end{answer}
% <SCPD_SUBMISSION_TAG>_1g
\clearpage

\LARGE
1.h
\normalsize

% <SCPD_SUBMISSION_TAG>_1h
\begin{answer}
  % ### START CODE HERE ###
  % ### END CODE HERE ###
\end{answer}
% <SCPD_SUBMISSION_TAG>_1h
\clearpage

\LARGE
2.a
\normalsize

% <SCPD_SUBMISSION_TAG>_2a
\begin{answer}
  % ### START CODE HERE ###
  % ### END CODE HERE ###
\end{answer}
% <SCPD_SUBMISSION_TAG>_2a
\clearpage

\LARGE
2.b
\normalsize

% <SCPD_SUBMISSION_TAG>_2b
\begin{answer}
  % ### START CODE HERE ###
  % ### END CODE HERE ###
\end{answer}
% <SCPD_SUBMISSION_TAG>_2b
\clearpage

\LARGE
2.c
\normalsize

% <SCPD_SUBMISSION_TAG>_2c
\begin{answer}
  The log-likelihood of an example $(x^{(i)}, y^{(i)})$ is defined as
  $\ell(\theta) = \log p(y^{(i)} \vert  x^{(i)}; \theta)$. To derive the stochastic
  gradient ascent rule, use the results in part (a) and the standard GLM
  assumption that $\eta = \theta^Tx$.
  \begin{flalign*}
    \frac{\partial \ell(\theta)}{\partial \theta_j}
    &= \frac{\partial \log p(y^{(i)} \vert  x^{(i)}; \theta)}{\partial \theta_j}\\
    &= \frac {\partial \log \left({\frac{1}{y^{(i)}!} \exp(\eta^T y^{(i)} -
    e^\eta)}\right)} {\partial \theta_j}\\
    &=\\
    % ### START CODE HERE ###
    % ### END CODE HERE ###
  \end{flalign*}

  Thus the stochastic gradient ascent update rule should be:

  \begin{equation*}
  \theta_j := \theta_j + \alpha \frac{\partial \ell(\theta)}{\partial \theta_j},
  \end{equation*}

  which reduces here to:
  % ### START CODE HERE ###
  % ### END CODE HERE ###
\end{answer}
% <SCPD_SUBMISSION_TAG>_2c
\clearpage

% <SCPD_SUBMISSION_TAG>_entire_submission

\end{document}
