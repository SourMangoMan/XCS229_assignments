% This contents of this file will be inserted into the _Solutions version of the
% output tex document.  Here's an example:

% If assignment with subquestion (1.a) requires a written response, you will
% find the following flag within this document: 📝_1a
% In this example, you would insert the LaTeX for your solution to (1.a) between
% the 📝_1a flags.  If you also constrain your answer between the
% START_CODE_HERE and END_CODE_HERE flags, your LaTeX will be styled as a
% solution within the final document.

% Please do not use the '📝' character anywhere within your code.  As expected,
% that will confuse the regular expressions we use to identify your solution.

% 📝_1a
\begin{answer}
  Since $g'(z) = g(z)(1-g(z))$ and $h(x) = g(\theta^T x)$, it follows that $\partial h(x) / \partial \theta_k = h(x)(1 - h(x)) x_k$.

  Letting $h_{\theta}(x^{(i)}) = g(\theta^T x^{(i)})
  = 1/(1 + \exp(-\theta^T x^{(i)})$, we have\\

  \begin{flalign*}
    \frac{\partial\log h_{\theta}(x^{(i)})}{\partial\theta_k} &= \\
    % ### START CODE HERE ###
    % ### END CODE HERE ###
    \frac{\partial\log(1 - h_{\theta}(x^{(i)}))}{\partial\theta_k} &= \\
    % ### START CODE HERE ###
    % ### END CODE HERE ###
  \end{flalign*}

  Substituting into our equation for $J(\theta)$, we have
  %
  \begin{flalign*}
    \frac{\partial J(\theta)}{\partial\theta_k} &=\\
    % ### START CODE HERE ###
    % ### END CODE HERE ###
  \end{flalign*}
  
  Consequently, the $(k, l)$ entry of the Hessian is given by
  
  \begin{flalign*}
    H_{kl} = \frac{\partial^2 J(\theta)}{\partial\theta_k\partial\theta_l} &=\\
    % ### START CODE HERE ###
    % ### END CODE HERE ###
  \end{flalign*}
  
  Using the fact that $X_{ij} = x_i x_j$ if and only if $X = xx^T$, we have
  
  \begin{flalign*}
    H &= \\
    % ### START CODE HERE ###
    % ### END CODE HERE ###
  \end{flalign*}

  To prove that $H$ is positive semi-definite, show $z^T Hz \ge 0$ for all $z\in\Re^\di$.
  
  \begin{flalign*}
    z^T H z &=\\
    % ### START CODE HERE ###
    % ### END CODE HERE ###
  \end{flalign*}
  
  % ### START CODE HERE ###
  % ### END CODE HERE ###
\end{answer}
% 📝_1a

% 📝_1c
\begin{answer}
  For shorthand, we let $\mc{H} = \{\phi, \Sigma, \mu_{0}, \mu_1\}$ denote
  the parameters for the problem.
  Since the given formulae are conditioned on $y$, use Bayes rule to get:
  \begin{align*}
    p(y =1\vert  x ; \mc{H}) &= \frac {p(x\vert y=1; \mc{H}) p(y=1; \mc{H})} {p(x; \mc{H})}\\
    & = \frac {p(x\vert y=1; \mc{H}) p(y=1; \mc{H})}
      {p(x\vert y=1; \mc{H}) p(y=1; \mc{H}) + p(x\vert y={0}; \mc{H}) p(y={0};
      \mc{H})}\\
    &=\\
    % ### START CODE HERE ###
    % ### END CODE HERE ###
  \end{align*}
\end{answer}
% 📝_1c

% 📝_1d
\begin{answer}
  First, derive the expression for the log-likelihood of the training data:
  \begin{flalign*}
    \ell(\phi, \mu_{0}, \mu_1, \Sigma) &= \log \prod_{i=1}^\nexp p(x^{(i)} \vert  y^{(i)}; \mu_{0}, \mu_1, \Sigma) p(y^{(i)}; \phi)\\
    &= \sum_{i=1}^{\nexp} \log p(x^{(i)} \vert  y^{(i)}; \mu_{0}, \mu_1, \Sigma) +
    \sum_{i=1}^{n} \log p(y^{(i)}; \phi)\\
    &=\\
    % ### START CODE HERE ###
    % ### END CODE HERE ###
  \end{flalign*}

  Now, the likelihood is maximized by setting the derivative (or gradient) with respect to each of the parameters to zero.\\

  \textbf{For $\mathbf{\phi}$:}

  \begin{flalign*}
    \frac{\partial \ell}{\partial \phi}&=\\
    % ### START CODE HERE ###
    % ### END CODE HERE ###
  \end{flalign*}

  Setting this equal to zero and solving for $\phi$ gives the maximum
  likelihood estimate.\\

  \textbf{For $\mathbf{\mu_0}$:}

  {\bf Hint:}  Remember that $\Sigma$ (and thus $\Sigma^{-1}$) is symmetric.

  \begin{flalign*}
    \nabla_{\mu_{0}}\ell &=\\
    % ### START CODE HERE ###
    % ### END CODE HERE ###
  \end{flalign*}

  Setting this gradient to zero gives the maximum likelihood estimate
  for $\mu_{0}$.\\

  \textbf{For $\mathbf{\mu_1}$:}

  {\bf Hint:}  Remember that $\Sigma$ (and thus $\Sigma^{-1}$) is symmetric.

  \begin{flalign*}
    \nabla_{\mu_{1}}\ell &=\\
    % ### START CODE HERE ###
    % ### END CODE HERE ###
  \end{flalign*}

  Setting this gradient to zero gives the maximum likelihood estimate
  for $\mu_{1}$.\\

  For $\Sigma$, we find the gradient with respect to $S = \Sigma^{-1}$ rather than $\Sigma$ just to simplify the derivation (note that $\vert S\vert  = \frac{1}{\vert \Sigma\vert }$).
  You should convince yourself that the maximum likelihood estimate $S_\nexp$ found in this way would correspond to the actual maximum likelihood estimate $\Sigma_\nexp$ as $S_\nexp^{-1} = \Sigma_\nexp$.

  {\bf Hint:}  You may need the following identities: 
  \begin{equation*}
    \nabla_S \vert S\vert  = \vert S\vert  (S^{-1})^T
  \end{equation*}
  \begin{equation*}
    \nabla_S b_i^T S b_i = \nabla_S tr \left( b_i^T S b_i \right) =
    \nabla_S tr \left( S b_i b_i^T \right) = b_i b_i^T
  \end{equation*}

  \begin{flalign*}
    \nabla_S\ell &=\\
    % ### START CODE HERE ###
  % ### END CODE HERE ###
  \end{flalign*}

  Next, substitute $\Sigma = S^{-1}$.  Setting this gradient to zero gives the required maximum likelihood estimate for $\Sigma$.\\
\end{answer}
% 📝_1d

% 📝_1f
\begin{answer}
  % ### START CODE HERE ###
  % ### END CODE HERE ###
\end{answer}
% 📝_1f

% 📝_1g
\begin{answer}
  % ### START CODE HERE ###
  % ### END CODE HERE ###
\end{answer}
% 📝_1g

% 📝_1h
\begin{answer}
  % ### START CODE HERE ###
  % ### END CODE HERE ###
\end{answer}
% 📝_1h

% 📝_2a
\begin{answer}
  % ### START CODE HERE ###
  % ### END CODE HERE ###
\end{answer}
% 📝_2a

% 📝_2b
\begin{answer}
  % ### START CODE HERE ###
  % ### END CODE HERE ###
\end{answer}
% 📝_2b

% 📝_2c
\begin{answer}
  The log-likelihood of an example $(x^{(i)}, y^{(i)})$ is defined as
  $\ell(\theta) = \log p(y^{(i)} \vert  x^{(i)}; \theta)$. To derive the stochastic
  gradient ascent rule, use the results in part (a) and the standard GLM
  assumption that $\eta = \theta^Tx$.
  \begin{flalign*}
    \frac{\partial \ell(\theta)}{\partial \theta_j}
    &= \frac{\partial \log p(y^{(i)} \vert  x^{(i)}; \theta)}{\partial \theta_j}\\
    &= \frac {\partial \log \left({\frac{1}{y^{(i)}!} \exp(\eta^T y^{(i)} -
    e^\eta)}\right)} {\partial \theta_j}\\
    &=\\
    % ### START CODE HERE ###
    % ### END CODE HERE ###
  \end{flalign*}

  Thus the stochastic gradient ascent update rule should be:

  \begin{equation*}
  \theta_j := \theta_j + \alpha \frac{\partial \ell(\theta)}{\partial \theta_j},
  \end{equation*}

  which reduces here to:
  % ### START CODE HERE ###
  % ### END CODE HERE ###
\end{answer}
% 📝_2c
